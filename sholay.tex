\documentclass{article}
\usepackage[utf8]{inputenc}
\usepackage{hyperref}

\title{Movie Review of\\\textsc{\LARGE{SHOLAY}}}
\date{April 2017}

\begin{document}

\maketitle

\section*{}

When it was first released in the mid-1970s, Sholay set Indian movie screens afire. It was the once-in-a-decade mega hit that smashed all records.

Even deep in the Hindi-hating southern state of Tamil Nadu, in the late 1970s one would come across young boys humming "Mehbooba, Mehbooba," a popular song from the movie.

The fast-paced film has a little something for everyone - gripping action, comedy, tragedy, romance and even some nice songs. 

Former inspector Thakur (Sanjeev Kumar) hires two criminals Veeru (Dharmendra) and Jai (Amitabh Bachchan) to catch a notorious dacoit Gabbar Singh (Amjad Khan) to avenge an old enmity. You see Gabbar had wiped out Thakur's family after escaping from jail.

Veeru and Jai arrive in Ramgarh village to help Thakur in his quest and that's when the rubber hits the road.

As the cruel dacoit Gabbar, Amjad Khan dazzles. Can anyone forget Gabbar's memorable lines such as "Tera kya hoga, Kalia"

Sanjeev Kumar shines as the angry ex-cop, Dharmendra excels and Amitabh is adequate. Hema Malini as the horse carriage driver Basanti and Jaya Bhaduri as Radha, the widowed daughter-in-law of Thakur, are good in supporting roles.

Director Ramesh Sippy delivers a knockout punch with Sholay.

Much of the movie was shot in the hills of Ramanagar, near Bangalore. Perhaps, that's why the village in which the movie is set was called Ramgarh.

Sholay was also reportedly the first 70-mm movie in India.

P.S: Alas, Amjad Khan, Sanjeev Kumar and R.D.Burman are no more

\section*{Sources}
\href{http://www.searchindia.com/search/bollywood-movies/sholay.html}{SearchIndia URL}

\section*{}
This section s only for demonstration purposes. Thankyou :)

\end{document}
